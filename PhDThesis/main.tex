% Here is a sample format for dissertation in math.  You need to check
% University of Pennsylvania Doctoral Dissertation Manual to adjust
% any changes they made at the webpage:

% http://www.upenn.edu/grad/DissManual.html.  

% This style file was used in connection with printing from printer 3one.
% (Different printers could give you different margins.)

% This sample can also be used for masters thesis, but you need to make
% some slight changes.
    
\documentclass[12pt,letterpaper]{report}
% XXX: FORMATTING:
% http://tex.stackexchange.com/questions/132170/what-do-headheight-headsep-etc-do-in-the-vmargin-package
\setlength{\hoffset}{0.5in}
\setlength{\voffset}{0in}
\setlength{\oddsidemargin}{0in}
\setlength{\topmargin}{0in}
\setlength{\headheight}{0in}
\setlength{\headsep}{0in}
\setlength{\textheight}{9in}
\setlength{\textwidth}{6in}
\setlength{\marginparsep}{0in}
\setlength{\marginparwidth}{0in}
\setlength{\marginparpush}{0in}
%\setlength{\footskip}{30pt}
\addtolength{\textheight}{-1\footskip}
\pagestyle{plain}

% bindingoffset=0.2in,footskip=.25in
%\usepackage[letterpaper,left=1.5in,right=1in,top=1in,bottom=1in,footskip=30pt]{geometry}
%%\setlength{\topmargin}{0in}
%\setlength{\headheight}{0in}
%%\setlength{\headsep}{0in}
%\setlength{\textheight}{8.5in}
%\setlength{\oddsidemargin}{0.55in}
%%\setlength{\marginparwidth}{0in}
%%\setlength{\marginparsep}{0in}
%%\setlength{\footskip}{0.3in}
%%\setlength{\evensidemargin}{-0.25in}
%\setlength{\textwidth}{5.87in}
%\setlength{\headsep}{-.19in}

%\newcommand{\doublespaced}{\renewcommand{\baselinestretch}{2}\normalfont}
%\newcommand{\singlespaced}{\renewcommand{\baselinestretch}{1}\normalfont}
%\newcommand{\draftspaced}{\singlespaced} %for draft only 
%\newcommand{\draftspaced}{\doublespaced} %for final version
%\newcommand{\subsubsubsection}{\paragraph}


% XXX: PACKAGES:
\usepackage[comma]{natbib}
\setcitestyle{square}
\usepackage[normalem]{ulem}	                        % underlining!
\usepackage[table, usenames,dvipsnames]{xcolor}            % color
\usepackage{extarrows}                              % http://ctan.org/pkg/extarrows
\usepackage{setspace}                               % \doublespacing or \onehalfspacing
%\usepackage{showframe}
%\usepackage{enumitem}

% Math
\usepackage{amsmath,amssymb,amsfonts,amsthm,dsfont} % math
\usepackage{algorithm,algorithmicx,listings}        % algorithms
\usepackage[noend]{algpseudocode}			        % necessary for algorithmicx

% Figures
\usepackage{graphicx}
\usepackage{tabularx}
\usepackage{multirow,multicol,rotating,diagbox}
\usepackage{booktabs}
\usepackage{makecell}
\usepackage[font={small}]{caption}   %onehalfspacing
\usepackage[font={small}]{subcaption}
\setlength{\belowcaptionskip}{-3.5pt}
\setlength{\abovecaptionskip}{3pt}
\captionsetup[algorithm]{font=small}
%\usepackage[svgnames]{xcolor}
%\usepackage[breaklinks=true, colorlinks=true,linkcolor=Blue, bookmarks=true, citecolor=Mahogany, urlcolor=RoyalPurple,linktoc=all]{hyperref}
\usepackage[breaklinks=true, colorlinks=false,linkcolor=black, bookmarks=true, citecolor=black, urlcolor=black,linktoc=all]{hyperref}


% XXX: COMMANDS:
\def\liminf{\mathop{\lim\inf}\limits}	% EXAMPLE: \liminf_n A_n
\def\limsup{\mathop{\lim\sup}\limits}	%
\def\argmin{\mathop{\arg\min}\limits}	%
\def\argmax{\mathop{\arg\max}\limits}	%
% Write above and below equal sign
\newcommand{\longeq}[2]{\xlongequal[\!#2\!]{\!#1\!}}
% #1 = top; #2 = bottom; #3 = inequality (<,>,\leq,\geq)
\newcommand{\longineq}[3]{\overset{#1}{\underset{#2}{#3}}}
\newcommand{\indicator}{\mathds{1}}
\DeclareMathOperator{\tr}{tr}
\DeclareMathOperator{\per}{\mathbf{per}}
\def\deg{^{\circ}}
\def\negquad{\mkern-18mu}             % negative quad space
\def\negqquad{\mkern-36mu}            % negative qquad space
\newcommand{\txbx}[1]{\boxed{\text{#1}}}
\newcommand{\TODO}[1]{{\color{red}#1}}
\newcommand{\tcn}[1]{\cellcolor{Magenta!#1!TealBlue}#1}% table colored number
\newcommand{\tcnb}[1]{\cellcolor{Magenta!#1!TealBlue}}% table colored number
\def\negl{\scalebox{0.75}{$\boldsymbol{\ominus}$}}
\def\posl{\scalebox{0.75}{$\boldsymbol{\oplus}$}}

\makeatletter
\newcommand{\customlabel}[2]{%
   \protected@write \@auxout {}{\string \newlabel {#1}{{#2}{\thepage}{#2}{#1}{}} }%
   \hypertarget{#1}{#2}
}
\makeatother

%\newtheorem{theorem}{Theorem}
%\newtheorem{proposition}{Proposition}
%\newtheorem{corollary}{Corollary}
%\newtheorem{definition}{Definition}
\newtheorem{assumption}{Assumption}[chapter]
\newtheorem*{assumption*}{Assumption}
%\newtheorem{remark}{Remark}
\newtheorem*{problem*}{Problem}
\newtheorem{problem}{Problem}
%\newtheorem{lemma}{Lemma} 


\newtheorem{theorem}{Theorem}[chapter]
\newtheorem{corollary}[theorem]{Corollary}
%\newtheorem*{main}{Main~Theorem}
\newtheorem{lemma}[theorem]{Lemma}
\newtheorem{proposition}[theorem]{Proposition}

\theoremstyle{definition}
\newtheorem{definition}{Definition}[chapter]
\newtheorem{property}{Property}[chapter]

\theoremstyle{remark}
\newtheorem*{remark*}{Remark}
\newtheorem{example}{Example}[chapter]

\numberwithin{equation}{chapter}


% NAMED THEOREMS
% EXAMPLE:
%\begin{namedthm}{Zorn's Lemma}[Zermelo]
%All well-behaved ordered sets have maximal elements.
%\end{namedthm}
\theoremstyle{plain} % just in case the style had changed
%\swapnumbers % optional, of course
\newcommand{\thistheoremname}{}
\newtheorem{genericthm}[theorem]{\thistheoremname}
\newenvironment{namedthm}[1]
  {\renewcommand{\thistheoremname}{#1}%
   \begin{genericthm}}
  {\end{genericthm}}
  
\newtheorem*{genericthm*}{\thistheoremname}
\newenvironment{namedthm*}[1]
  {\renewcommand{\thistheoremname}{#1}%
   \begin{genericthm*}}
  {\end{genericthm*}}


\usepackage[per-mode=symbol]{siunitx}
\usepackage{paralist}

%\doublespaced
\def\thetitle{DATA-DRIVEN MODELING, CONTROL AND TOOLS FOR CYBER-PHYSICAL ENERGY SYSTEMS}
\def\theauthor{Madhur Behl}
\def\theyear{2015}




%====================================================================
\hypersetup{
  pdfauthor={\theauthor},%
  pdftitle={\thetitle},%
  pdfsubject={PhD dissertation},%
  pdfkeywords={cyber-physical systems, data-driven, modeling, white-box, grey-box, uncertainty propagation, demand response, electricity, real-time pricing, energy, buildings, model predictive control, regression trees, non-parametric statistics, control synthesis, tool, load curtailment, energy systems, energy analytics, sensor placement, model accuracy }
  %pdfproducer={LaTeX},%
  %pdfcreator={pdfLaTeX}
}

%\addtocontents{toc}{\protect{\pdfbookmark[0]{\contentsname}{toc}}}



\begin{document}
\pagenumbering{roman}

%=================================================
%%% TITLE PAGE
\newpage
\phantomsection
\addcontentsline{toc}{chapter}{Title}
\thispagestyle{empty}
\vspace*{\fill}
\begin{center}
\thetitle

\vspace*{0.3in}
\theauthor

\vspace*{0.3in}
A DISSERTATION\\
$ $\\
in \\
$ $\\
Computer and Infomation Science\\
$ $\\
Presented to the Faculties of the University of Pennsylvania\\
in Partial Fulfillment of the Requirements for the\\
Degree of Doctor of Philosophy\\
$ $\\
\theyear
\end{center}

\vspace{0.6 in}
\noindent\makebox[0in][l]{\rule[2ex]{3in}{.3mm}}
Rahul Mangharam, Associate Professor of Electrical and Systems Engineering\\Supervisor of Dissertation

\vspace*{0.6 in}
\noindent\makebox[0in][l]{\rule[2ex]{3in}{.3mm}}
Rajeev Alur, Zisman Professor of Computer and Information Science\\Graduate Group Chairperson

\vspace*{0.2in}
\noindent Dissertation Committee:\\
Insup Lee, Cecilia Fitler Moore Professor of Computer and Information Science\\
Pieter J. Mosterman, Adjunct Professor at the School of Computer Science at McGill University\\
Richard Gray, Biomedical Engineer at FDA
\vspace*{\fill}
%=================================================




%=================================================
%%% COPYRIGHT PAGE
\newpage
\thispagestyle{empty}
\vspace*{\fill}
\begin{center}
From Verified Models to Verified Code for Safe Medical Devices

\vspace*{0.6 in}
COPYRIGHT

\vspace*{0.6 in}
\theyear

\vspace*{0.6 in}
\theauthor
\end{center}
\vspace*{\fill}
%=================================================


%=================================================
%%% DEDICATION
\newpage
\begin{center}
\vspace*{\fill}
\it{To the universe.}
\vspace*{\fill}
\end{center}
%=================================================


%=================================================
%%% ACKNOWLEDGEMENTS
\newpage
\phantomsection
\addcontentsline{toc}{chapter}{Acknowledgments}
\chapter*{Acknowledgments}
%\input{acknowledgements.tex}
Thank you, God bless you; and may God bless the United States of America
%=================================================




%=================================================
%%% ABSTRACT
\newpage
\phantomsection
\addcontentsline{toc}{chapter}{Abstract}
\begin{center}
ABSTRACT\\
$ $\\
\thetitle\\
$ $\\
Zhihao Jiang\\
Rahul Mangharam\\
\end{center}

\noindent Energy systems are experiencing a gradual but substantial change in moving away from being non-interactive and manually-controlled systems to utilizing tight integration of both cyber (computation, communications, and control) and physical representations guided by first principles based models, at all scales and levels.  
Furthermore, peak power reduction programs like demand response (DR) are becoming increasingly important as the volatility on the grid continues to increase due to regulation, integration of renewables and extreme weather conditions.
In order to shield themselves from the risk of price volatility, end-user electricity consumers must monitor electricity prices and be flexible in the ways they choose to use electricity. 

This requires the use of control-oriented predictive models of an energy system’s dynamics and energy consumption. Such models are needed for understanding and improving the overall energy efficiency and operating costs. 
However, learning dynamical models using grey/white box approaches is very cost and time prohibitive since it often requires significant financial investments in retrofitting the system with several sensors and hiring domain experts for building the model.
We present the use of data-driven methods for making model capture easy and efficient for cyber-physical energy systems. 

We develop Model-IQ, a methodology for analysis of uncertainty propagation for building inverse modeling and controls. 
Given a grey-box model structure and real input data from a temporary set of sensors, Model-IQ evaluates the effect of the uncertainty propagation from sensor data to model accuracy and to closed-loop control performance.
We also developed a statistical method to quantify the bias in the sensor measurement and to determine near optimal sensor placement and density for accurate data collection for model training and control.
Using a real building test-bed, we show how performing an uncertainty analysis can reveal trends about inverse model accuracy and control performance, which can be used to make informed decisions about sensor requirements and data accuracy.

We also present DR-Advisor, a data-driven demand response recommender system for the building's facilities manager which provides suitable control actions to meet the desired load curtailment while maintaining operations and maximizing the economic reward.
 We develop a model based control with regression trees algorithm (mbCRT), which allows us to perform closed-loop control for DR strategy synthesis for large commercial buildings. 




 
%=================================================



%=================================================
%%% CONTENTS
\newpage
\phantomsection
%\setcounter{tocdepth}{2}
\addcontentsline{toc}{chapter}{Contents}
\tableofcontents
\clearpage
\phantomsection
\addcontentsline{toc}{chapter}{\listtablename}
\listoftables
\clearpage
\phantomsection
\addcontentsline{toc}{chapter}{\listfigurename}
%\addtocontents{toc}{\protect{\pdfbookmark[0]{\listfigurename}{lof}}}
\listoffigures
%=================================================




%=================================================
%%% MAIN
\newpage
\pagenumbering{arabic}
\setcitestyle{aysep={}} % no separation between author and year


%\include{chapters/introduction/introduction}
%\include{chapters/modeling/buildingmodeling}
%\include{chapters/modeliq/modeliq}
%\include{chapters/trees/rtree}
%\include{chapters/dradvisor/dradvisor}
%\include{chapters/analytics/energy}
%\chapter{Discussion and Open Challenges}
Closed-loop medical devices like implantable cardiac devices have both diagnostic and therapeutic capabilities and interact with the patient autonomously in closed-loop. Their autonomy makes them among the highest risk devices which require the most stringent regulation. Currently, clinical trials are the primary means to identify risks associated with the closed-loop interaction between the devices and the patient. While such clinical trials are a necessity they are expensive and ineffective for verification of the safety and efficacy of medical device software. 

Model-based design can potentially enable closed-loop evaluation earlier in the design process. This approach requires validated physiological models that represents the closed-loop interaction of different physiological conditions from the device's perspective. In this effort, we use an implantable pacemaker as an example to demonstrate the application of model-based design in providing safety and effectiveness towards ``regulatory grade evidence" of the device and describe how these activities align into the regulatory process. 

We developed heart models that capture the electrical behaviors across a range of heart conditions, and tailored the heart models for closed-loop model checking and closed-loop testing, which have different requirements. In closed-loop model checking, an abstract model of the pacemaker was validated against physiological requirements. We identified the need for heart models at different abstraction levels and demonstrated an automated approach to select the most appropriate heart model for specific safety requirements. The abstract model of the pacemaker was then automatically translated to Stateflow chart using a model translation tool and then generated into C code implementation. With this model-driven design, we are able to retain the safety properties of the modeled device from verified models to verified code. In closed-loop testing, we use more refined heart models to capture mechanisms that were not captured in the abstract heart models and evaluated pacemaker algorithms. 

%As implantable medical devices grow in sophistication with significant capabilities implemented in software, firmware bugs account for an increasing fraction of device recalls. Since the FDA  does not test, verify or certify the software in such devices, there is an urgent need for a systematic methodology to guarantee the safety of the device software. Furthermore, this safety must account for the closed-loop interaction with the organs of interest. In this effort, we have presented a first step in the direction of a holistic approach for model-based testing and physical patient--device interactive validation. We achieve this through the design of an integrated functional and formal model of the heart and pacemaker device using timed automata. Using this closed-loop system, we conducted formal verification and translate the models to Stateflow for simulation-based testing and then automatically generate code from the verified models. We investigated basic operations, complex algorithms such as response to Pacemaker Mediated Tachycardia and platform-based testing of physical issues such as crosstalk and lead displacement. 

%To validate the environment model, we described an Abstraction Tree approach to select the heart model most appropriate for the specific requirement. This approach separated the evaluation phases where domain knowledge was necessary and where it was not, which simplified the verification of the medical device across a range of heart conditions. The model-based design toolchain from verified models to verified code is conducted within the clinically-relevant context and captures the complexities of interaction with physiological components. 

%This effort describes early steps toward cyber-physical model based testing, validation and verification of medical cyber-physical systems. This is a challenging domain as patient modeling is both complex, highly variable and non-deterministic and the safety properties must include over-approximated models for verification, abstract models for simulation and be realizable in physical form for testing. We aim to extend the current efforts to evaluate more complex devices such as implantable cardioverter defibrillators (ICDs) which are rate-adaptive and operate with varying levels of hysteresis. 

Eventually we aim to conduct \emph{Model-based Clinical Trials} with automated approaches to capture and tune patient-specific electrophysiological heart models using data acquired from ablation procedures. By applying parametric and sensitivity analysis to a small sampling of real patient heart models, across a select set of cardiac conditions, to derive a statistically significant, and physiologically relevant, population of patient models. This allows us to explore a wider range of heart behaviors and expose more corner cases to isolate software safety issues prior to an actual clinical trial. Using certified heart models, a model-based clinical trial provides additional confidence in the closed-loop safety and efficacy of medical devices prior to randomized controlled clinical trials. Model-based clinical trials for medical device software have the potential to complement the current regulatory approach by reducing the cost, scope and probability of failure of a traditional clinical trials. The area of Medical Cyber-Physical Systems is in its early days with several exciting and urgent fundamental challenges in modeling, control, verification and testing for higher confidence life-critical systems.





%\include{tex/Conclusion}


%\appendix
%\include{tex/Appendix}
%=================================================




%=================================================
%%% BIBLIOGRAPHY
\newpage
%\clearpage
\phantomsection
\addcontentsline{toc}{chapter}{Bibliography}

\bibliographystyle{plainnat}
\bibliography{bibliography}

%=================================================

\end{document}

%\bibliographystyle{amsplain}
%\bibliographystyle{plainnat}

